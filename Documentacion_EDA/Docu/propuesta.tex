\documentclass[letterpaper]{article}
\usepackage[utf8]{inputenc}
\usepackage[T1]{fontenc}
\usepackage[activeacute,english]{babel}
\usepackage[vmargin=4cm,tmargin=3cm,hmargin=2cm,letterpaper]{geometry}%
\usepackage{helvet}
\usepackage{amsmath,amsfonts,amssymb}
\usepackage{graphicx}
\usepackage{color}
\usepackage{xcolor}
\usepackage{verbatim}
\usepackage{tabls}
\usepackage{lastpage}
\usepackage{fancyhdr}
\usepackage{url}
\usepackage{listings}
\usepackage{tikz}
\usepackage{pgf}
\usepackage{pgffor}
\usepgfmodule{plot}
\usepackage{wrapfig}
\usepackage{ifpdf}
\usepackage{amssymb}
\usepackage{pifont}
\usepackage{epstopdf}
\usepackage{graphicx} % Allows including images
\usepackage{booktabs} % Allows the use of \toprule, \midrule and \bottomrule in tables
\usepackage{tikz}
\usetikzlibrary{arrows,decorations,snakes,backgrounds,fit,calc,through,scopes,positioning,automata,chains,er,fadings,calendar,matrix,mindmap,folding,patterns,petri,plothandlers,plotmarks,shadows,shapes,shapes.arrows,topaths,trees}

\lstset{% general command to set parameter(s)
%   basicstyle=\small,
  % print whole listing small
%   keywordstyle=\color{black}\bfseries\underbar,
  % underlined bold black keywords
%   identifierstyle=,
  % nothing happens
%   commentstyle=\color{white}, % white comments
%   stringstyle=\ttfamily,
  % typewriter type for strings
  showstringspaces=false}
  % no special string spaces

\pagestyle{fancy}
\color{black}
\fancyhead{}
\renewcommand{\headrule}{\hrule\vspace*{0.5mm}\rule{\linewidth}{0.8mm}}
\renewcommand{\familydefault}{\sfdefault}

\graphicspath{{./images/}}
\lhead{\includegraphics[width=2cm]{logoucr.png}}
\rhead{\includegraphics[width=3cm]{eie-text-gray-6x3cm.png}}
\chead{UNIVERSIDAD DE COSTA RICA\\FACULTAD DE INGENIERÍA\\ESCUELA DE INGENIERÍA ELÉCTRICA\\\textbf{ESTRUCTURAS ABSTRACTAS DE DATOS Y\\ ALGORITMOS PARA INGENIERÍA}\\IE-0217\\I CICLO 2014\\PROPUESTA DEL PROYECTO DE ESTRUCTURAS DE DATOS}

\lfoot{}%
\cfoot{}%
%\cfoot{\thepage\ de \pageref{LastPage}}%
\rfoot{}%

%%%%%%%%%%%%%%%%%%%%%%%%%%%%%%%%%%%%%%%%%%%%%%%%%%%%%%%%%%%%%%%%%%%%%%%%%%%%%%%%%%%%%%%%%%%%%%%%%%%%%%%%%%%%%%%
\newcommand{\uic}{blue} %user-input color
%%%%%%%%%%%%%%%%%%%%%%%%%%%%%%%%%%%%%%%%%%%%%%%%%%%%%%%%%%%%%%%%%%%%%%%%%%%%%%%%%%%%%%%%%%%%%%%%%%%%%%%%%%%%%%%%%%
\newcommand{\uim}{\_\_} %user-input marker
%%%%%%%%%%%%%%%%%%%%%%%%%%%%%%%%%%%%%%%%%%%%%%%%%%%%%%%%%%%%%%%%%%%%%%%%%%%%%%%%%%%%%%%%%%%%%%%%%%%%%%%%%%%%%%%%%%
\newcommand{\userinput}[1]{\textcolor{\uic}{\uim#1\uim}}


%%%%%%%%%%%%%%%%%%%%%%%%%%%%%%%%%%%%%%%%%%%%%%%%%%%%%%%%%%%%%%%%%%%%%%%%%%%%%%%%%%%%%%%%%%%%%%%%%%%%%%%%%%%%%%%%%%
\begin{document}\vspace*{2cm}
%%%%%%%%%%%%%%%%%%%%%%%%%%%%%%%%%%%%%%%%%%%%%%%%%%%%%%%%%%%%%%%%%%%%%%%%%%%%%%%%%%%%%%%%%%%%%%%%%%%%%%%%%%%%%%%%%%

%%%%%%%%%%%%%%%%%%%%%%%%%%%%%%%%%%%%%%%%%%%%%%%%%%%%%%%%%%%%%%%%%%%%%%%%%%%%%%%%%%%%%%%%%%%%%%%%%%%%%%%%%%%%%%%%%%
\begin{center}
\Huge
\textbf{Data Structures Implementations for K-Da Library}
\vspace*{1cm}
\end{center}

\noindent
\small\baselineskip=14pt
\textbf{Estudiantes:} \\
\text{David Pérez Bolaños - B04769}\\
\text{Andrey Pérez Salazar - B25084}\\
\text{Andrés Sánchez López - B26214}\\


\begin{figure}[ht]
\includegraphics[width=1\linewidth]{ese.jpg}
\end{figure}


%%%%%%%%%%%%%%%%%%%%%%%%%%%%%%%%%%%%%%%%%%%%%%%%%%%%%%%%%%%%%%%%%%%%%%%%%%%%%%%%%%%%%%%%%%%%%%%%%%%%%%%%%%%%%%%%%%
\section{Introducción}
\hyphenation{fun-cio-na-li-dad}
La creación de librerías en lenguajes de programación ayuda a generar una interfaz bien definida para una cierta funcionalidad en específico, 
estas sirven para separar por módulos un programa, y así generar un código más claro y ordenado.\\

Una librería es un conjunto de funciones para desarrollar software, por lo general no son programas, pero si son utilizadas por los programas 
para poder funcionar de forma correcta; el desarrollo de librerías sirve como apoyo para los programadores a tener más facilidades de implementación 
en sus programas y a contar con más recursos para realizar sus proyectos.\\

En este caso, queremos implementar una librería sobre manipulación y análisis de objetos tridimensionales
en el lenguaje C++.\\

La idea  de esta librería es crear métodos o funciones para la manipulación y representación 
de cualquier objeto real, de manera virtual; esta técnica de pasar de un objeto real a representarlo de manera virtual, se utiliza mucho 
en videojuegos, efectos especiales, medicina, simuladores, entre otros.\\

Para nuestro caso utilizaremos para representar los objetos de manera virtual, la técnica de malla de triángulos en 3D, en dónde primero debemos 
representar por medio de una nube de puntos el objeto y luego formar los triangulos con cada uno de esos puntos; para ello utilizaremos el kinect, 
obteniendo así los datos y luego realizar los algoritmos necesarios para la ejecución de las funciones de comparación de datos, generando así la librería.\\


Una segunda parte del proyecto, consiste en la implementación de algunas estructuras de datos para la manipulación de información obtenida mediante el kinect, de manera que se quiere realizar un análisis implementando algunas de ellas, para así determinar y comparar, qué estructuras son las más eficientes a la hora de manipular esta información.\\

En general, las estructuras de datos implementarán las funciones básicas de acceso o búsqueda de datos, lo cual nos ayudará a tener un manejo más rápido y ordenado de los datos obtenidos por el kinect.

 

%%%%%%%%%%%%%%%%%%%%%%%%%%%%%%%%%%%%%%%%%%%%%%%%%%%%%%%%%%%%%%%%%%%%%%%%%%%%%%%%%%%%%%%%%%%%%%%%%%%%%%%%%%%%%%%%%%
\section{Desarrollo}

\subsection{K-Da Library}

Como se mencionó con anterioridad, la creación de librerías de programación corresponden a una funcionalidad en específico. En nuestro caso, la librería
implementada corresponde a un conjunto de métodos computacionales para llevar a cabo la comparación de movimientos. Para lograr esto, se implementó 
la librería en lenguaje C++, sin embargo, los datos de los movimientos son tomados de un controlador de juego libre y entretenimiento llamado Kinect; creado por Alex Kipman, desarrollado por Microsoft para la videoconsola Xbox 360.(http://personales.alumno.upv.es/alrafua/asignaturas/SES/Perifericos/Interfaces_humanas/camaras_reconocimiento.html)
Además se usa también el lenguaje de programación Processing para poder obtener del Kinect los datos que se produzcan. Relacionadas estas tres importantes
herramientas es que nuestra librería da el funcionamiento esperado. 

\subsection{Funcionamiento de K-Da Library}

Retomando las herramientas que forman parte de la librería, iniciamos este apartado para describir, de una manera general, como se da la comparación de los movimientos.

\subsubsection{Código en C++}

El código de la librería fue desarrollado en el lenguaje C++. Este código presenta tres diferentes clases, una encargada de recibir los datos de movimiento que provienen del Kinect, sin embargo, es importante mencionar que, en nuestro caso, los movimientos captados son únicamente realizados por humanos, es decir, sí se presenta un movimiento de algún objeto no humano el Kinect no tomará los datos de tal movimiento,
esto ya que está dentro del propio desarrollo del Kinect esta utilizar técnicas de reconocimiento de voz y reconocimiento facial para la identificación automática de los usuarios. (http://malenyabrego.wordpress.com/2011/12/31/mundo-kinect/) Otra de las clases presentes en el código se encarga de convertir los datos (esto último se explicará con mucho más detalle más adelante) que fueron 
recibidos del Kinect y es aquí donde toma participación la última de las clases que es la encargada de realizar la comparación de los movimientos; cabe decir que para realizar una comparación es necesario contar con mínimo dos movimientos.

\subsubsection{Processing}

El lenguaje de programación Processing es la herramienta que utilizamos para tomar los datos que el Kinect genere según el movimiento y convertir estos datos en archivos \textit{.txt} para luego estos, ser utilizados dentro del código en C++. En el código de Processing se pueden elegir 
los movimientos de los que se quieren obtener los datos. Es decir, si necesitamos solo los datos de movimiento que genere el Kinect solo en el \textit{joint} de la muñeca y no de todos los \textit{joints} del diagrama en general se puede realizar el cambio dentro del código. En nuestro caso, 
Processing reproduce los datos del movimiento de los \textit{joints} del torso, cuello, hombro y muñeca. 




\subsection{Objetivos Específicos}

Los objetivos específicos son:\\

\begin{enumerate}
\item Ampliar las funciones de la librería K-Da, de modo que se puedan hacer una mayor cantidad de comparaciones de movimientos.
\item Investigar sobre algunas estructuras de datos que puedan mejorar el rendimiento o eficacia de la informacíón utilizada 
en algoritmos de la librería K-Da.
\item Implementar algunas de las estructuras de datos en la librería K-Da para determinar cuáles dan mejores resultados.
\end{enumerate}


%%%%%%%%%%%%%%%%%%%%%%%%%%%%%%%%%%%%%%%%% YO %%%%%%%%%%%%%%%%%%%%%%%%%%%%%%%%%%%%%%%%%%




\section{Objetivos}

\subsection{Objetivo General}


El objetivo general consiste en analizar algunas de las diferentes estructuras de datos y su desempeño en términos de eficiencia en la librería K-Da
realizada en el primer proyecto del curso.

\subsection{Objetivos Específicos}

Los objetivos específicos son:\\

\begin{enumerate}
\item Ampliar las funciones de la librería K-Da, de modo que se puedan hacer una mayor cantidad de comparaciones de movimientos.
\item Investigar sobre algunas estructuras de datos que puedan mejorar el rendimiento o eficacia de la informacíón utilizada 
en algoritmos de la librería K-Da.
\item Implementar algunas de las estructuras de datos en la librería K-Da para determinar cuáles dan mejores resultados.
\end{enumerate}

%%%%%%%%%%%%%%%%%%%%%%%%%%%%%%%%%%%%%%%%%%%%%%%%%%%%%%%%%%%%%%%%%%%%%%%%%%%%%%%%%%%%%%%%%%%%%%%%%%%%%%%%%%%%%%%%%%
\section{Metodología}


Se realizará la investigación sobre estructuras de datos que se necesiten para la manipulación de objetos tridimensionales. Previamente, se revisarán los algoritmos encontrados, para determinar la necesidad de nuestro proyecto.  Una vez hecho esto, se buscará implementar varias estructuras de datos para poder compararlas entre si a la hora de organizar la información de los arrays provenientes del kinect.\\

Lo que se realizará primero, es una investigación sobre que estructuras de datos son utilizadas generalmente para este tipo de proyectos, esto con el fin de entender la utilización de las estructuras de datos en los algoritmos de comparación. Para lograr esto, se leerá toda la bibliografía necesaria para poder entender de la mejor manera que es lo que se necesita exactamente. Una vez entendido algunas de las diferentes implementaciones ya existentes, se procederá a organizar la información proveniente del kinect con las estructuras de datos encontradas 
y así compararlas una a una.\\

Por lo tanto será necesario también, la utilización de un kinect para obtener datos de diferentes figuras tridimensionales. Finalmente, se buscará determinar la estructura de datos más eficiente para que la librería en c++ creada anteriormente funcione de la manera más eficiente posible.\\

%%%%%%%%%%%%%%%%%%%%%%%%%%%%%%%%%%%%%%%%%%%%%%%%%%%%%%%%%%%%%%%%%%%%%%%%%%%%%%%%%%%%%%%%%%%%%%%%%%%%%%%%%%%%%%%%%%


\section{Cronograma}

\begin{center}
\begin{tabular}{l l   @{\hspace{1cm}}p{10cm}}
\cline{3-3}

\toprule
\textbf{Semana} & \textbf{Fechas} & \textbf{Actividad} \\
\midrule
1 & 1 a 7 de junio & Estudio de las posibles estructuras de datos a utilizar en el proyecto. \\
2 & 8 al 14 de junio & Implementación de las diferentes estructuras de datos escogidas. \\

3 & 15 al 21 de junio & Análisis de eficiencia de las estructuras de datos escogidas. \\
4 & 22 al 28 de junio & Determinación de la mejor estructura de datos y realización del informe.\\

\bottomrule
\end{tabular}
\end{center}

\section{Referencias}

\begin{enumerate}

\item Richard, J. Computer Science Division. University of California at Berkeley. Triangle. A Two-Dimensional Quality Mesh Generator and 
Delaunay Triangulator. Encontrado el 13 de abril del 2014 en: http://www.cs.cmu.edu/~quake/triangle.html
\item Escenografía Intermedial. Nuevos medios y tecnologías afines a la escena. (15 de mayo del 2012).
Nube de puntos (Point Cloud) con Kinect. Encontrado el 13 de abril del 2014 en: http://escenografiaaumentada.wordpress.com/2012/05/15/148/
\item OPENKINECT. Encontrado el 13 de abril del 2014 en: http://openkinect.org/wiki/Main\_Page
\item Joyanes, L., Sánchez, L. \& Zahonero, I. (2007). Estructuras de datos en C++ (1ra ed.) Madrid: McGraw-Hill / Interamericana de España, S.A.



\end{enumerate}

	
\end{document}

